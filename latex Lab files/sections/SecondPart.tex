\section{Second Member}
This is the section dedicated to one of the team members, and it should be written individually . It can include a range of things; first subsection is a space for you to point out the strengths and weaknesses of the module, including complaints about the module coordinator Max Wilson. The second section should have a selfie image with Max! The last part of it is the most important one. You will need to write a paragraph about what you have learned in this module. You can write it in \textbf{Bold} if you want or you can use other fonts. 

Please do not forget:
\begin{itemize}
	\item First paragraph should have your comments about the module
	\item Second one, a selfie img with Max
	\item Last one, what you learned in this module.
\end{itemize}

\subsection{Comments about the module}
This module has been a different take on what we've seen so far in Computer Science. It's shown us a different take on real life applications of creating software for clients and has highlighted how working in a team is imperative to large or even small scale software creation, driving many people out of their comfort zones and forcing them to adapt to new types of tasks aside from simply coding.

\subsection{Selfie with Max}

To include an image, you will need to remove the comments from the code below, place an image in the main fo lder, and do not forget to put the name of the image instead of ImgName. 

\begin{figure}[h]
	\caption{Selfie with Max}
	\centering
	\includegraphics[width=0.5\textwidth]{maxnme}
	\label{fig:selfie}
\end{figure}

You can then use the label of the figure to reference it later with the command ${\backslash}ref.$ you can comment out the next line to see an example of how it works.

 My selfie with Max is in  Figure~\ref{fig:selfie}.

\subsection{What I have learned in this module}
I have learned new skills involved less in coding, but more in the surrounding areas of creating software, such as determining clients needs and meeting their expectations. I have also learned how to use new programs such as Visual Paradigm to create models and diagrams with UML. I now understand that making software isn't necessarily about just coding but involves many stages that makes making software a lengthy and deeply constructed process. Another aspect learned is working with other people (of whom I have never met before) which will be an important aspect of creating any software.
